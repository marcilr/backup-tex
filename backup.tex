%% -*- Mode: LaTeX -*-
%%
%% backup.tex
%% Copyright 2012 by Raymond E. Marcil
%% Created Sun Nov 18 18:53:42 AKST 2012
%% 
%% Datestamped Backups Implemented with Hardlinks and Rsync
%%

  %%
%%%%%% Preamble.
  %%

%% Specify DVIPS driver used by things like hyperref
\documentclass[12pt,letterpaper,dvips]{article}

\usepackage{svninfo}
\usepackage{fullpage}
\usepackage{fancyvrb} 
\usepackage{graphicx}
\usepackage{figsize}
\usepackage{calc}
\usepackage{xspace}
\usepackage{booktabs}
\usepackage[first,bottomafter]{draftcopy}
\usepackage[numbib]{tocbibind}

\usepackage{amssymb}              %% AMS Symbols, used for \checkmark
\usepackage{multicol}


%%
%% Hyperref package for embedding URLs for clickable links in PDFs, 
%% also specify PDF attributes here.
%%
%% FIXME: Blue hyperlinks would be nice.  Not certain which option actually does this.
%%
%% The pdfborder={0 0 0} is what ellimated the blue box around the url displayed by \href{}{}.
%% The command pdfborder={0 0 1} would display a box with thickness of 1 pt.
%%
%% Hypertext marks in LATEX: a manual for hyperref
%% by Sebastian Rahtz and Heiko Oberdiek - November 2012
%% http://ctan.org/pkg/hyperref 
%% http://mirror.hmc.edu/ctan/macros/latex/contrib/hyperref/doc/manual.html
%%
\usepackage[
colorlinks=false,
hyperindex=false,
urlcolor=blue,
pdfborder={0 0 0},
pdfauthor={Raymond E. Marcil},
pdftitle={Datestamped Backups using Hardlinks and Rsync},
pdfcreator={ps2pdf},
pdfsubject={Datestamped Backups using Hardlinks and Rsync},
pdfkeywords={backups, datestamped, hardlinks, rsync}
]{hyperref}

%%
%% Extract SVN metadata for use elsewhere.
%% This information has:
%% o the filename
%% o the revision number
%% o the date and time of the last Subversion co command
%% o name of the user who has done the action
%%
\usepackage{svninfo}
\svnInfo $Id$


  %%
%%%%%% Customization.
  %%

% On letter paper with 10pt font the Verbatim environment has 65 columns.
% With 12pt font the environment has 62 columns.  Exceeding this will exceed
% the frame and will look ugly.  YHBW.  HAND.
\RecustomVerbatimEnvironment{Verbatim}{Verbatim}{frame=single}

\renewenvironment{description}
                 {\list{}{\labelwidth 0pt \iteminden-\leftmargin
                          \let\labelsep\hsize
                          \let\makelabel\descriptionlabel}}
                 {\endlist}
\renewcommand*\descriptionlabel[1]{\hspace\labelsep\sffamily\bfseries #1}


  %%
%%%%%% Commands.
  %%

\newcommand{\FIXME}[1]{\textsf{[FIXME: #1]}}
\newcommand{\cmd}[1]{\texttt{#1}}


%% Squeeze space above/below captions
\setlength{\abovecaptionskip}{4pt}   % 0.5cm as an example
\setlength{\belowcaptionskip}{4pt}   % 0.5cm as an example


%% Tex really adds a lot of whitespace to itemized 
%% lists so define a new command itemize* with a 
%% lot less whitespace.  Found this in the British
%% Tex faq.
\newenvironment{itemize*}%
  {\begin{itemize}%
    \setlength{\itemsep}{0pt}%
    \setlength{\parsep}{0pt}}%
  {\end{itemize}}

%% Squeeze space
\renewcommand\floatpagefraction{.9}
\renewcommand\topfraction{.9}
\renewcommand\bottomfraction{.9}
\renewcommand\textfraction{.1}   
\setcounter{totalnumber}{50}
\setcounter{topnumber}{50}
\setcounter{bottomnumber}{50}


  %%
%%%%%% Document.
  %%

\title{Datestamped Backups\\
       Implemented with\\
       Hardlinks and Rsync}

\author{Raymond E. Marcil\\
        \texttt{marcilr@gmail.com}
}

% Display subversion revision and date.
\date{Revision \svnInfoRevision
      \hspace{2pt}
      (\svnInfoLongDate)}

\begin{document}

\maketitle

\begin{abstract}
  This document contains the design of a script to generated datestamped backups using hardlinks
  and rsync.  The script will generate backups for every day for the
  past week, every month for the past year, and every prior year.

\end{abstract}

\vspace{2.0in}

%% Draw DNR logo and address at bottom of page
%%\begin{figure}[h]
%%        \hspace{0.32in}
%%        \SetFigLayout{1}{1}
%%        \begin{minipage}[b]{0.16\figwidth}
%%                \includegraphics[width=\textwidth]{dnr_bwlogo.eps}
%%        \end{minipage}
%%        \hspace{5pt}
%%        \begin{minipage}[b]{\figwidth}
%%                \bf{Alaska Department of Natural Resources}\\
%%                \small{\sf{Division of Support Services\\
%%                Land Records Information Section\\
%%                550 W. 7th Ave. Suite 706\\
%%                Anchorage, Alaska 99501}}
%%        \end{minipage}
%%\end{figure}


\newpage


\tableofcontents

\newpage
\listoffigures
\listoftables


\newpage

%% =============== List of Abbreviations ===============
%% =============== List of Abbreviations ===============
\section{List of Definitions and Abbreviations}
\begin{itemize*}
  \item{\begin{bf}rsync\end{bf}} - A fast, versatile, remote (and local) file-copying tool.
\end{itemize*}


%% =================== Introduction ====================
%% =================== Introduction ====================
\newpage
\section{Introduction}
The goal is to create a script that will generate backups for every day for the past
week, every month for the past year, and every prior year.



%% ======================== Examples =============================
%% ======================== Examples =============================
%% ======================== Examples =============================
\newpage
\section{Examples}
Series of useful \LaTeX\ markup. Need to break out to 
separate examples.tex file.

\subsection{Escaping $<$ and $>$ Symbols}
To get \$$<$\$ or \$$>$\$ just wrap the symbols in \$ for math mode.

\subsection{Enumerate}
\begin{enumerate}
  \item{DNR} - Alaska State Department of Natural Resources
    \begin{itemize*}
      \item{HI} - Historical Index, not maintained since 1982
      \item{LE} - Land Estate, maintained by SGU
      \item{ME} - Mineral Estate, maintaind by SGU
   \end{itemize*}

  \item{Alaska State Surveys}
    \begin{itemize*}
      \item{ASBLT} - As-Built Survey
      \item{ASCS} - Cadastral Survey
    \end{itemize*}
\end{enumerate}


%% ======================= Comments =======================
%% ======================= Comments =======================
\subsection{Comments}
\begin{center}
\framebox{
\begin{minipage}[t]{0.9\textwidth}
\cmd{COMMENTS} Comment --- \emph{Sean Weems, Spring 2003}\\
We should get the \cmd{COMMENTS} column searchable via the 
landrecords application before we do much anything else -- shouldn't
be too hard.
\end{minipage}
}
\end{center}

\begin{center}
\framebox{
\begin{minipage}[t]{0.9\textwidth}
\emph{Errata: Plats spanning multiple sections}\\
A few anomalies can be observed in the \cmd{AKPLATS}
table. Specifically plats exist that span multiple sections. 
Since the table only has a single column, \cmd{SCODE}, 
that accepts a single section code, SGU (Status
Graphics Unit) has handled this problem by entering multiple 
rows in the table, each with a different section that point to
the same plat or file. Multiple section plats are indicated by  setting 
the \cmd{TCODE} column to the value \cmd{37}, and making an 
appropriate notation like \emph{Section 24-25-26-27} in the 
\cmd{REMARKS} column.\\
\FIXME{Perhaps the \cmd{SCODE} column should accept an array of sections?}

\end{minipage}
}
\end{center}

%% ====================== Footnotes ======================
%% ====================== Footnotes ======================
\subsection{Footnotes}
 Some footnotes here
\ref{fig:uss-index} for an example. Yet another
\ref{table:uss-index} example.

%% ================ Simple Table Examples ================
%% ================ Simple Table Examples ================
\newpage
\subsection{Simple Table Examples}
\begin{table}[htb]
\begin{tabular}{|p{.25\textwidth}|p{.20\textwidth}|p{.47\textwidth}|}\hline 
Column Name&Type&Description\\ 
\hline
EQS&VARCHAR2(1)&!NULL map shows village selections\\
ITM\_COL&VARCHAR2(1)&USGS ITM column: 1-6\\
ITM\_ROW&VARCHAR2(1)&USGS ITM row: A-E\\
QMQ\_ABBR\_DNR&VARCHAR2(3)&Three character DNR abbreviation for the QMQ\\
RASTER\_FILENAME&VARCHAR2(50)&Physical path to file\\
RASTER\_PATHNAME&VARCHAR2(50)&URL path to PDF of map\\
SCODE&VARCHAR2(2)&Supplement map code: 1,2,3,...\\
COMMENTS&VARCHAR2(256)&Plat comments\\
\hline
\end{tabular}
\caption {\cmd{EASEMENTS\_17B} Table}
\label{table:easements_17b}
\end{table}

%% A simple table example
%% The htb attribute attempts to inline the table or figure
%% where you put it in the document
\begin{table}[htb]
\begin{center}
\begin{tabular*}{\textwidth}{@{}p{.25\textwidth}@{}p{.75\textwidth}}
\hline
\hline\\[-2.5ex]
XML element&Descripton\\
\hline
\hline\\[-1.5ex]   %% Trick to add whitespace after horizontal line
FNUM&US Survey file number\\ 
MERIDIAN&BLM meridian code\\
&\hspace{10pt}12 = Copper River\\
&\hspace{10pt}13 = Fairbanks\\
&\hspace{10pt}28 = Seward\\
&\hspace{10pt}44 = Kateel\\
&\hspace{10pt}45 = Umiat\\
TOWNSHIP&Five character Township code\\
RANGE&Five character Range code\\
PAGE&Survey page number 1,2,3,...\\
FILENAME&Relative path to file in direcory\\[1.5ex]
\hline
\end{tabular*}
\caption {USS XML index elements}
\label{table:uss-index}
\end{center}
\end{table}


%% ==================== Another Simple Table Example =================
%% ==================== Another Simple Table Example =================
\subsection{Another Simple Table Example}
\begin{table}[htb]
\begin{tabular}{|p{.16\textwidth}|p{.17\textwidth}|p{.62\textwidth}|}\hline 
Column Name&Type&Description\\ 
\hline
MTR&VARCHAR2(9)&Meridian, Township, Range, example: \emph{C026S054E}\\
QMQ&VARCHAR2(3)&Quarter Million Quadrangle code,\\
&&\hspace{10pt}example: \emph{DIL} (Dillingham quadrangle)\\
\hline
\end{tabular}
\caption {\cmd{XREF\_MTR\_QMQ} Table}
\label{table:xref_mtr_qmq}
\end{table}

%% ====================== Verbatim ==========================
%% ====================== Verbatim ==========================
\newpage
\subsection{Verbatim}

\begin{center}
\begin{figure}
\begin{quote}
\small
\begin{Verbatim}[frame=none]
gis/raster/
  dnr/
    map_library/
    plats/
      SP/YYYYMMDD/*.pdf               # indexed
      HI/YYYYMMDD/*.pdf               # Indexed
      ASLS/YYYYMMDD/*.pdf             # Indexed
    recorded-plats/
      YYYYMMDD/*.pdf
  blm/
    easements_17b/YYYYMMDD/*.pdf      # indexed
    mtp/YYYYMMDD/*.pdf                # non-indexed
    usrs/YYYYMMDD/*.pdf               # indexed
    usrs-notes/YYYYMMDD/*.pdf         # indexed
    uss/YYYYMMDD/*.pdf                # indexed
    uss-notes/YYYYMMDD/*.pdf          # indexed
    usms/YYYYMMDD/*.pdf               # indexed
    usms-notes/YYYYMMDD/*.pdf         # indexed
  usgs/
    drg/
      collared/ 
        250K/
        63K/
        25K/
        24/
      decollared/
      tools/
      missing\_data/
    dem/
    doq/
    topo/
\end{Verbatim}
\normalsize
\end{quote}
\caption{File and Directory Structure}
\label{fig:dirlayout}
\end{figure}
\end{center}


%% ======================== Appendix =============================
%% ======================== Appendix =============================

\clearpage
\newpage
\section*{Appendix}

\subsection*{Links}
\href{http://www.linuxjournal.com/article/8590}{Build a Home Terabyte Backup System Using Linux}\\
by Duncan Napier - Nov 29, 2005\\
\href{http://www.linuxjournal.com/article/8590}{http://www.linuxjournal.com/article/8590}\\

%%
%% The \noindent alleviates the paragraph indentation.
%%
\noindent \href{http://www.mikerubel.org/computers/rsync_snapshots/}{Easy Automated Snapshot-Style Backups with Linux and Rsync by Mike Rubel.}\\
\href{http://www.mikerubel.org/computers/rsync_snapshots/}{http://www.mikerubel.org/computers/rsync\_snapshots/}\\

%%\url{http://www.google.com}{google}

\end{document}

%% Local Variables:
%% fill-column: 78
%% mode: auto-fill
%% compile-command: "make"
%% End:
